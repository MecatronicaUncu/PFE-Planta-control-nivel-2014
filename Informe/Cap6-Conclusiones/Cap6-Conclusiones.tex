\chapter{Conclusiones}
\label{ch:conclusiones}

\section{Conclusión técnica}
\label{sec:ConclusionTecnica}

Al finalizar este \gls{pfe}, pudo verificarse el cumplimiento de las
especificaciones definidas en la Sec. \ref{subsec:especificaciones}.
El funcionamiento de la planta es correcto, manteniendo el objetivo pedagógico
para el que fue construida.

Los elementos que conforman la planta de nivel son los mismos que se
encuentran en la industria.
Se realizó un estudio detallado de cada uno de ellos en el Cap.
\ref{ch:DisenoEnsamblado}.
Se verifica además el costo inferior a una solución llave en mano, como se
mostró en la Sec. \ref{sec:AnalisisEconomico}.

Se validó el correcto funcionamiento de la planta escribiendo un programa de
ejemplo para el \gls{plc}, de control de nivel.
Además, se diseñó una interfaz hombre-máquina mediante software de tipo
\gls{scada}.
Se comprobó un correcto control regulatorio, llegando rápidamente a la consigna
con pocas oscilaciones (ver \ref{subsec:ZiegerNichols}).

Todos los elementos de la planta fueron correctamente identificados en un
\gls{pyid}.
Por otra parte, el conexionado fue consignado en esquemas eléctricos.
El proyecto incluyó el diseño de circuitos electrónicos con su correspondiente
tarjeta (regulador de tensión, conversor \verb|RS-485| a \verb|RS-232C|), que
quedan a disposición para futuros proyectos.

Los Caps. \ref{ch:progPLC} y \ref{ch:scada} fueron escritos con el
objetivo de poder ser replicados por parte de otros alumnos.
Finalmente, se escribieron manuales de puesta en marcha y operación de la
planta.

\section{Perspectivas}
\label{sec:Perspectivas}
Este informe se ha enfocado principalmente en los aspectos constructivos
de la planta.
No obstante, otras experiencias y trabajos se pueden realizar. Se puede citar:
\begin{itemize}
\item Se instaló una placa orificio (ver Sec. \ref{subsec:PlacaOrificio}), 
utilizada
para medir el caudal de ingreso al tanque cuyo nivel se pretende controlar. Por 
lo que podría plantearse un esquema de control avanzado tipo maestro-esclavo, 
utilizando otro lazo \gls{pid} del \gls{plc}, siendo éste esclavo del maestro
de nivel.
\item La determinación de los parámetros del controlador se realizó por 
el método de Ziegler–Nichols(ver Sec. \ref{subsec:ZiegerNichols}) a lazo 
cerrado, podría hacerse a lazo abierto y cotejar los parámetros del lazo de 
nivel con ambos métodos.
\end{itemize}

\section{Conclusiones personales}
\label{sec:ConclusionPersonal}


El desarrollo del proyecto final es una etapa fundamental en la formación del
Ingeniero en Mecatrónica, ya que permite realizar una síntesis de los
conocimientos adquiridos durante la formación.
Durante el proyecto se aplicaron conceptos de control de
procesos, mecánica de fluidos y máquinas hidráulicas, electrotecnia,
electrónica, autómatas programables, diseño de interfaces hombre-máquina, por 
poner algunos ejemplos.

Pero además se nos permitió diseñar, construir y validar un proyecto completo,
desde su origen hasta su fin.
Obtenemos de esta manera una visión integral del proceso de concepción de un
sistema de control para un proceso determinado.
Estos conceptos pueden ser aplicados en muchas industrias del medio, donde el
control de procesos tiene un rol fundamental: petroleo y gas, vitivinícola y
alimentaria en general, producción de energía, química, entre otras disciplinas.

Asimismo, consideramos muy importante haber participado en el desarrollo de un
proyecto que será utilizado luego por otros estudiantes.
Creemos que la cooperación entre los alumnos y los profesores es fundamental
para lograr un desarrollo óptimo de la carrera en la facultad.

Finalmente, este proyecto tiene una gran importancia a nivel personal para cada
uno de nosotros.
Con él finaliza un ciclo de arduo trabajo durante nuestra
carrera de grado.
