\chapter{Tablero Eléctrico}
\label{ch:tablero}

Luego del diseño y ensamblado, se procedió a la instalación
y cableado de los elementos eléctricos y electrónicos de la planta. 
Por simplicidad, se dividirá este capítulo en 
dos secciones\todo{Foto o diagrama del tablero}:
\begin{itemize}
 \item \textbf{Potencia:} Corresponde a la instalación y el cableado
 de los elementos eléctricos de la planta, destinados a alimentar los motores de 
 las bombas y los circuitos lógicos.
 \item \textbf{Señal:} Corresponde a la instalación y el cableado
 de los elementos electrónicos (\gls{plc}, módulos inalámbricos
 sensores de presión, caudal y actuadores), 
 cuyo objetivo es controlar la planta.
\end{itemize}

\section{Potencia}
\label{sec:Potencia}
Tal como se describió previamente, los elementos de potencia deben 
\textbf{alimentar} los motores y elementos lógicos de la planta.
Se optó por trabajar con corriente alterna monofásica ($220\,V\,-\,50\,Hz$), 
debido a que es la alimentación de las bombas\todo[color=blue!40]{Ref Schema!}.

\subsection{Descripción de los elementos}
\subsubsection{Corte General}
El primer elemento que encontramos en el tablero es el interruptor 
termomagnético de corte general\todo[color=blue!40]{Ref Schema!}
con número de modelo $C60 N$.
La función del termomagnético es proteger la instalación frente a posibles
cortocircuitos, y ser utilizado para encender o apagar la planta.
\subsection{Alimentación de los Motores de las Bombas}
Para cada una de las bombas, se utilizarán los siguientes elementos de 
protección y comando:
\begin{itemize}
 \item \textbf{Relés:} Reciben una  señal de $24\,V$ de parte del PLC que activa
 su bobina. En salida, tenemos una señal de $220\,V$ que se conecta con los 
 contactores previo paso por el relevo térmico\todo{figuraLaddder?}.
 \item \textbf{Relevo Térmico:} Tiene como objetivo evitar la sobrecorriente en
 los motores. Para el proyecto, se configuró el relevo térmico con una corriente
 de $1.6\,A$. En caso de superar esta corriente, interrumpe la línea de señal
 del contactor y detiene los motores.
 \item \textbf{Contactores:} Enciende los motores, a partir de la señal de 
 $220\,V$ del Relé, que pasa por el relevo térmico. Se utilizaron 
contactores \verb|LC1-D09| de Schneider Electic, cuya hoja de datos 
puede encontrarse en el anexo \todo{anexo contactor}.Debido a que los 
contactores
 son trifásicos, debió realizarse un bucle con dos de las fases (ver diagrama
 \todo[color=blue!40]{Ref Schema!}).
\end{itemize}

En la figura \ref{fig:diagramaLadderContactor} se muestra el diagrama
simplificado de conexión. Se observa que  el motor funcionará mientras la bobina
del relé esta activada y mientras el relevo térmico no acuse sobrecorriente.

\begin{figure}
 \centering
 \missingfigure[figwidth=6cm]{Ladder motores}
 \caption{Esquema Ladder del conexionado de los motores}
 \label{fig:diagramaLadderContactor}
\end{figure}

\subsection{Fuente de alimentación}
Los circuitos lógicos de la planta funcionan con una tensión de $24\,V$, de
corriente continua. Dado que en la planta sólo contamos con tensión alterna de
la red, necesitamos una fuente de alimentación para entregar la tensión
requerida en el \gls{plc}. Se optó por utilizar una fuente de alimentación
industrial, montada sobre un riel DIN. Las características técnicas son las
siguientes:
\begin{center}
\begin{tabular}{|l|l|}
\hline
Marca & ReignPower\\
Tensión Entrada& $100-120$ / $200-240\,V$\\
Tensión Salida& $24\,V$\\
Corriente Salida Maxima& $4.2\,A$\\
Potencia & $100\,W$\\
\hline
\end{tabular}
\end{center}

\section{Señal}
\label{sec:Senal}

El circuito de señal tiene como objetivo relevar variables de la planta (caudal,
presión), procesarlas (\gls{plc}) y entregar señales de control a la válvula
neumática. Además, debe conectarse el \gls{plc} a la computadora de control
mediante el \gls{scada}\todo{Esta bien? Scada debería aparecer antes}.
Se tratará en primer medida el \gls{plc}, encargado de la adquisición (a 
la través de las DP cell), procesamiento, actuación (mediante la válvula 
electroneumática) y control de los motores de las bombas. Luego, se detallarán 
los módulos inalámbricos utilizados para comunicar el \gls{plc} con la 
computadora de control.

\subsection{PLC}
\label{subsec:plc}

Para automatizar la planta, se optó por utilizar un \gls{plc}, que es 
un sistema electrónico digital de automatización y control, pensado para 
aplicaciones industriales (seguro frente a condiciones adversas, vibraciones, 
ruido electromagnético) \cite{bib:ApuntesJGabriel}\todo{poner Libro Control}.

Un \gls{plc} está conformado por los siguientes elementos:
\begin{itemize}
 \item \textbf{Procesador:} En función de las señales de los sensores y de la 
información almacenada, genera señales hacia los actuadores. El procesador 
sigue un algoritmo, denominado programa de aplicación, que debe ser programado 
previamente. Entre los lenguajes de programación definidos en la norma 
\verb|IEC 61131| podemos citar: Ladder, Texto Estructurado, Lista de 
Instrucciones y GRAFCET (SFC).
 \item \textbf{I/O discretas:} las entradas y salidas discretas pueden ser 
utilizadas para acciones de contol discretas, tal como encender motores o leer 
sensores discretos (fin de carrera, por ejemplo). En nuestro proyecto, se 
utilizan las salidas discretas \verb|Q0.0| y \verb|Q0.1| para activar los relés 
de los motores. Además \verb|I0.0| e \verb|I0.1| se utilizan como 
enclavamiento: conectadas a los contactores, verifican que el motor se 
encuentre encendido.
\item \textbf{I/O analógicas:} Mediante una entrada analógica, se pueden leer 
señales de corriente ($4-20\,mA$) o tensión ($0-10\,V$) provenientes de los 
sensores. De la misma manera, las salidas analógicas permiten enviar señales de 
tensión o corriente. En nuestro proyecto, las entradas analógicas\todo{por qué 
es IW0.1.0?} \verb|IW0.1.0| e \verb|IW0.1.1| permiten leer los valores de los 
DP cell de caudal y nivel, respectivamente\todo{verificar orden, también en 
gráfico}. Se emplea la salida analógica \verb|IW1.0| para controlar la válvula 
neumática\todo{Verificad iw1.0}. Cabe destacar que nuestro \gls{plc} 
no posee I/O analógicas, y se obtuvieron anexando un módulo \verb|TWD AMM 6HT|.
\item \textbf{Comunicación:} El puerto de comunicaciones permite programar el 
controlador, como así también enviar y recibir comandos durante la ejecución 
mediante el \gls{scada}. En el \gls{plc} utilizado contamos con un puerto de 
comunicaciones 
\verb|RS 485|.
\end{itemize}

\subsection{Módulos inalámbricos}
\label{subsec:inalambrico}

Para comunicar la planta con la computadora de control, sin necesidad de 
utilizar un cable de par trenzado, se decidió utilizar un módulo inalámbrico
\verb|ADC-200| de CTM Electrónica. Se trata de un equipo de comunicación para 
realizar enlaces inalámbricos que utilicen la interfaz \verb|RS 232| o 
\verb|RS 485|. Podemos citar, entre las características técnicas relevantes:

\begin{center}
\begin{tabular}{|l|l|}
\hline
Alimentación & $5\,VCC$\\
Consumo& $60\,mA$\\
Alcance& $1000\,m$\\
Baud Rate &2400 a 19200 bps \\
Frecuencia& 431 a 470 Mhz mas de 100 canales\\
\hline
\end{tabular}
\end{center}

Para el proyecto, se configuró el módulo tal como lo indica el 
anexo\todo{anexo}, estipulando una velocidad de transferencia de 
\todo{velocidad}\verb|xxxx bps|, sin pariedad\todo{verificar}. Además, se 
cableó el módulo al \gls{plc} tal como lo explicita el anexo\todo{anexo}.

Cabe destacar que el módulo inalámbrico \textbf{no puede utilizarse} para 
la programación del \gls{plc}, debido a que el software (Twido) acusa un error 
de timeout. No obstante, el funcionamiento es correcto para recibir informacion 
y enviar consignas mediante el \gls{scada}.