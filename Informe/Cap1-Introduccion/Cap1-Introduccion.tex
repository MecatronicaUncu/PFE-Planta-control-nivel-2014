\chapter{Introducción}
\label{ch:intro}

\section{Descripción del problema}
\label{sec:DescripcionProblema}
El objetivo de este trabajo es el desarrollo de una planta 
educativa de control de nivel
(\textit{Level control trainer}).
La planta será utilizada para realizar prácticas con los
alumnos, perimitiendo aplicar los conceptos de 
control de procesos, programación de \gls{plc}, SCADA,
Ajuste de lazos PID, Teoría de Cascadas. 
Además, el alumno tendrá la aproximación a elementos de adquisición que 
se encuentran en la industria, tales como sensores diferenciales de presión
(\textit{DP cell}) y placas orificio.
El proyecto está destinado a ser utilizado por las cátedras de
Automatismos Industriales, Autómatas y Control Discreto 
(Ingeniería en Mecatrónica)
como así también Instrumentación y Control Automático )Ingeniería Industrial
y de Petróleos)\todo{verificar}.

En la actualidad, en el Laboratorio de Control Automático de la Facultad, 
se disponen con equipos similares al descripto\todo{citar equipos?}.
No obstante, el proyecto planteado tiene la característica de ser 
\textbf{móvil}: se encuentra montado en una estructura que 
permite su desplazamiento, y el enlace a la 
computadora de control se realizará de manera inalámbrica.
% Además, se utilizará una válvula neumática...

Se encuentran en el mercado plantas similares, llave en mano. 
Dos firmas del medio fueron consultadas, para conocer 
el precio en el mercado de un desarrollo similar:
\begin{itemize}
 \item \textbf{Reino Unido}: Una planta con características similares
 a la descripta tiene un precio de $16225\,USD$ en puerto de Buenos Aires
 \item \textbf{Italia}: Con un precio de $28403\,EUR$ en puerto. 
 El modelo propuesto por la firma italiana
 posee un precio superior debido a que está construído en acero inoxidable.
\end{itemize}

Se observa que, debido al elevado costo del producto en el mercado,
se justifica su desarrollo  y construcción
en la facultad.
Se plantea entonces el desafío de diseñar, ensamblar y programar una planta
móvil de control de nivel en 
el Laboratorio de Control Automático de la Facultad.

%Por qué se hizo el trabajo? Comparación con otras plantas
%(plantas para comprar Y plantas que puglesi ya tiene)

\subsection{Especificaciones}

Las siguientes especificaciones deben cumplirse para la nueva
planta:

\begin{itemize}
\item Debe cumplir su objetivo pedagógico, permitiendo 
 a los alumnos comprender los principios del control automático.
 \item Su costo debe ser inferior a una solución llave en mano.
 \item Deberán utilizarse actuadores, elementos de adquisición
 y controladores que se encuentren en plantas industriales.
 \item Se confeccionará un manual de uso sencillo, para 
 fácil comprensión por parte de los alumnos.
\end{itemize}
\todo{Mejorar}

\section{Materiales disponibles}
\label{sec:MaterialesDisponibles}

\section{Solución propuesta}
\label{sec:SolucionPropuesta}

\section{Organización}
\label{sec:Organizacion}
\subsection{División del trabajo}
\subsection{Planning}
