\chapter{Manual del Usuario}
\label{anexo:manualUsuario}

% A continuación se presentan los pasos necesarios para la
% correcta utilización de la planta.

\section{Verificaciones Iniciales}
\label{anexo:verificaciones}
Antes de poner en funcionamiento la planta verifique cada uno de los siguientes 
items:


\begin{tcolorbox}[title=Nivel de agua]
Verifique que cada tanque no tenga más del 55 \% (aprox.) del total de su 
capacidad. 
\end {tcolorbox}
\begin{lattention}
En modo automático, si el nivel del tanque controlado desciende por debajo del 
20 \% se detiene la planta. 
\end{lattention}



\begin{tcolorbox}[title=Válvulas manuales, breakable]
La planta cuenta con 5 válvulas manuales, se describirá la posición de cada una 
de ellas para un correcto funcionamiento de la planta. 
 
 \begin{itemize}
  \item Vaciado de los tanques: Completamente cerradas
  \item Control de caudal del tanque de reserva:
  La apertura de la válvula debe ser tal que el manómetro que esta junto
  a la misma marque $0.5\frac{Kg}{cm^2}$. Esto nos asegura que el sistema está 
  equilibrado.
  \item Tiempo muerto: Abierto, De tal manera que no tenga efecto sobre la 
  planta la manguera del tiempo muerto.
  \item Perturbación bomba llenado: Totalmente cerrada, para asegurar que el 
  tanque controlado no se descargue más rápido de lo que se llena.
  \item Perturbación bomba vaciado: Totalmente abierta.
 \end{itemize}
 \tcblower
 Estas válvulas están presentes para modificar el sistema y generar distintos 
escenarios en la planta. Al modificar las características del sistema se deben 
ajustar los valores del controlador.
\end {tcolorbox}


\begin{tcolorbox}[title=Presión de aire]
  Asegúrese que la presión de aire este regulada a 4 bares. Se debe controlar 
  el correcto funcionamiento del filtro y regulador. 
 \tcblower
  Un filtro regulador como el presente en la planta se utiliza de la siguiente 
  manera: 
 \begin{enumerate}
    \item Tener precaución de desconectar la entrada a la válvula. 
    \item Se conecta el aire al regulador. 
    \item Se visualiza el valor de presión en el manómetro
    \item Se desplaza hacia arriba la maneta de regulación y se gira hasta 
      alcanzar el valor de presión deseado.
    \item Se vuelve a su posición original la maneta de regulación.
 \end{enumerate}
\end {tcolorbox}
%\todo{agregar filtro de aire}

\begin{lattention}
Nunca conectar la linea de aire comprimido sin el filtro y  
regulador de manera directa a la válvula.
\end{lattention}


\begin{tcolorbox}[title=Motores]
Para poder activar los motores necesitamos estar seguros que no están trabados.
 Para ello antes de comenzar con la actividad sobre la planta se recomienda 
retirar la jaula que recubre los ventiladores de los motores y hacer girar los 
mismos para destrabarlos.
\end {tcolorbox}


\begin{lattention}
Luego de haber realizado todas las verificaciones se debe conectar a la red 
eléctrica 220 V, corriente alterna.
\end{lattention}


 
 \section{Activación Manual de los Motores}
  Se explica a continuación los pasos a seguir para activar de modo manual los 
motores.
 \begin{lattention}
 Exclusivamente en el caso de que el sistema SCADA no responda 
 %Para ciertos casos en los que puede llegar a ser% 
 es necesario la activación manual de los motores.
 %, como por ejemplo podría ser el exceso de nivel en 
 %uno de los tanques, problemas con el Software SCADA, etc.
 Por favor, en caso de dudas consulte con el encargado del laboratorio.
\end{lattention}

%   \begin{enumerate}
%    \item Controlar que no estén trabados.
%    \item Energizar el sistema.
%    \item Conectar el aire, presión 4 bares.
%    \item Activar el relé manualmente que sea necesario.
%   \end{enumerate}

\begin{table}[H]
\centering
\renewcommand*{\arraystretch}{0.01}
\begin{tabular}{*{2}{m{0.45\textwidth}}}
% \hline
%   Controlar que las bombas no estén bloqueadas.
%   &\begin{center}
%     \rule{0.4\textwidth}{0.3\textwidth}
%     %\includegraphics[width=0.4\textwidth]
%     %  {Cap5-SCADA/images/startUp.jpeg}
%   \end{center}\\
\hline
    Conectar la planta a la linea de presión de aire. Presión 4 bares
    &\begin{center}
      \rule{0.4\textwidth}{0.3\textwidth}
      %\includegraphics[width=0.4\textwidth]
	%{Cap5-SCADA/images/database.jpeg}
    \end{center}\\
\hline
    Energizar el sistema. Activar en interruptor termomagnético ubicado dentro 
    del tablero eléctrico
    &\begin{center}
      \rule{0.4\textwidth}{0.3\textwidth}
      %\includegraphics[width=0.4\textwidth]
	%{Cap5-SCADA/images/database1.jpeg}
    \end{center}\\
\hline
    Activar el relé manualmente lo que sea necesario. En la imagen se puede 
    apreciar la manija que se debe accionar para activar el 
    relé.
    &\begin{center}
      \rule{0.4\textwidth}{0.3\textwidth}
      %\includegraphics[width=0.4\textwidth]
	%{Cap5-SCADA/images/database1.jpeg}
    \end{center}\\
\hline
\end{tabular}
\end{table}

\todo{referencia a la hoja de datos del relé}
