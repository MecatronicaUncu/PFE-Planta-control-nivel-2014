\section{Puesta en Marcha de la planta}
\label{anexo:puestaEnMarcha}

En el siguiente anexo se presentan las condiciones necesarias para el
correcto funcionamiento de la planta.

\begin{itemize}
 \item Energía
 
 Se debe conectar a la red eléctrica 220 V, corriente alterna.
 
 \item Nivel de agua
 
 Se recomienda que cada tanque no tenga más del 55 \% del total de su 
 capacidad. Es importante destacar que si el nivel del tanque desciende
 debajo del 20 \% se detiene la planta. 
 
 \item Válvulas manuales
 
 La planta cuenta con 5 Válvulas manuales, se describirá la posición de 
 cada una de ellas para un correcto funcionamiento de la planta. De todas
 maneras cabe recordar que estas válvulas están presentes para modificar el 
 sistema y generar una situación de la planta distinta, la que para ser 
 controlada deben cambiarse los valores del controlador.
 
 \begin{itemize}
  \item Vaciado de los tanques: Completamente cerradas
  \item Control de caudal del tanque de reserva:
  La apertura de la válvula debe ser tal que el manómetro que esta junto
  a la misma marque tantos bares. Esto nos asegura que el sistema está 
  equilibrado.
  \item Tiempo muerto:
  \item Perturbación bomba llenado:
  \item Perturbación bomba vaciado:
 \end{itemize}

 \item Presión de aire
 
 La presión de aire debe ser de 4 bares. Se debe controlar el correcto 
 funcionamiento del filtro y regulador. Nunca conectar el aire sin el filtro y 
 regulador.
 
 \end{itemize}

 \subsection{Problemas}
 
 \begin{itemize}
  \item Exceso de nivel de agua en un tanque.
  
  Es una situación en la que la solución es pasar fluido desde
  un tanque al otro. Para ello se debe abrir la válvula y encender la bomba
  correcta para traspasar fluido de un reservorio al otro.
  
  Se presentan dos soluciones posibles.
  
  \begin{itemize}
   \item Desde el Software \gls{scada}:
 
   \item Manualmente:
   
   Se pueden activar manualmente los contactores de manera manual.
 
 \end{itemize}

  \item 
  
 \end{itemize}
