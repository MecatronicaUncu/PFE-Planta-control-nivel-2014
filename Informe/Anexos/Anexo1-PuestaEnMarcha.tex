\section{Puesta en Marcha de la planta}
\label{anexo:puestaEnMarcha}

En el siguiente anexo se presentan las condiciones necesarias para el
correcto funcionamiento de la planta.

\begin{itemize}
 \item Energía
 
 Se debe conectar a la red eléctrica 220 V, corriente alterna.
 
 \item Nivel de agua
 
 Se recomienda que cada tanque no tenga más del 55 \% del total de su 
 capacidad. Es importante destacar que si el nivel del tanque desciende
 debajo del 20 \% se detiene la planta. 
 
 \item Válvulas manuales
 
 La planta cuenta con 5 Válvulas manuales, se describirá la posición de 
 cada una de ellas para un correcto funcionamiento de la planta. De todas
 maneras cabe recordar que estas válvulas están presentes para modificar el 
 sistema y generar una situación de la planta distinta, la que para ser 
 controlada deben cambiarse los valores del controlador.
 
 \begin{itemize}
  \item Vaciado de los tanques: Completamente cerradas
  \item Control de caudal del tanque de reserva:
  La apertura de la válvula debe ser tal que el manómetro que esta junto
  a la misma marque tantos bares. Esto nos asegura que el sistema está 
  equilibrado.
  \item Tiempo muerto:
  \item Perturbación bomba llenado:
  \item Perturbación bomba vaciado:
 \end{itemize}

 \item Presión de aire
 
 La presión de aire debe ser de 4 bares. Se debe controlar el correcto 
 funcionamiento del filtro y regulador. Nunca conectar el aire sin el filtro y 
 regulador.
 
 Un filtro regulador como el presente en la planta se utiliza de la siguiente manera:
 
 \begin{enumerate}
  \item Se conecta el aire. Tener precaución de desconectar la entrada a la válvula. 
  \item Se visualiza el manómetro, sobre el mismo hay una perilla.
  \item Se desplaza hacia arriba la maneta de regulación y se gira hasta alcanzar el
  valor de presión deseado.
  \item Se vuelve a su posición original la maneta de regulación.
 \end{enumerate}

 
 \todo{Referencia a la imagen del filtro}
 
 \item Motores
 
 Para poder activar los motores necesitamos estar seguros que no están trabados.
 Para ello antes de comenzar con la actividad sobre la planta se recomienda retirar
 la jaula que recubre los ventiladores de los motores y hacer girar los mismos
 para destrabarlos.
 
 \end{itemize}

 
 \section{Activación de los motores}
 
 Para ciertos casos en los que puede llegar a ser necesario la activación 
 manual de los motores, como por ejemplo podría ser el exceso de nivel en 
 uno de los tanques, problemas con el Software \gls{scada}, etc. se explican
 a continuación los pasos a seguir para activar de modo manual los motores.

  \begin{enumerate}
   \item Controlar que no estén trabados.
   \item Energizar el sistema.
   \item Conectar el aire, presión 4 bares.
   \item Activar el relé manualmente que sea necesario.
  \end{enumerate}
  
  En la imagen se puede apreciar la manija que se debe accionar para activar 
  el relé.
\todo{referencia a la hoja de datos del relé}
