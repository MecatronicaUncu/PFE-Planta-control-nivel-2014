\chapter{Diseño y Ensamblado}
\label{ch:DisenoEnsamblado}
En esta sección intentaremos describir el camino realizado para la concreción del proyecto, desde 
las primeras instancias con la confección e interpretación del diagrama de cañerías e instrumentación, 
luego con la construcción de la planta a partir de los elementos disponibles, haciendo mención de 
las decisiones trascendentes para el trabajo.

\section{P \& I D}
\label{sec:p&id}

\subsection{Diagrama de cañerías e instrumentación}
Un diagrama de cañerías e instrumentación es un esquema de la planta que nos brinda una idea 
general del problema que debemos abordar. En él se encuentra los elementos que componen el proceso. 
Estos diagramas emplean círculos y símbolos para representar cada elemento y como están interconectados, 
los elementos presentes en este esquema están normalizados por la \emph{Standard S5.1
Instrumentation Symbol Specification}.

Cada elemento debe ir acompañado de una denominación la que presenta la siguiente codificación:


\begin{itemize}  
 \item Primer letra: 
 Es usada para designar la variable medida. Puede ser :
 \begin{itemize}
  \item Presión
  \item Nivel
  \item Caudal
  \item Temperatura
 \end{itemize}

 \item Letras siguientes:
 Son usadas para designar la función del componente, o para modificar el sentido de la primer letra.
 Pueden ser:
 \begin{itemize}
  \item Indicador
  \item Almacenaje de datos
  \item Controlador
  \item Transmisor
 \end{itemize}
\end{itemize}

\subsection{P \& I D de nuestra planta}
De acuerdo a los componentes detallados en el capitulo 1 de este informe se realizó el esquema P\&ID
de la planta de control de nivel. Para la confección de la misma se utilizó una plantilla disponible
en Google Docs, la decisión sobre la misma se hizo dado que era suficiente para la complejidad de
nuestro proyecto.
\todo{Colocamos la imagen acá o hacemos una referencia al anexo?}

\section{Estructura de Soporte}
\label{sec:EstructuraSoporte}

Dado que el objetivo principal del proyecto es hacer un elemento para fines educativos, la estructura
debía ser móvil, dotada de cuatro ruedas con frenos de seguridad que permitan mantener fija la estructura
al momento de la demostración de la planta.

La estructura es de caño estructural \todo{imagen de la estructura} se puede apreciar en figura(tanto).
Sus dimensiones son las siguientes:

 \begin{itemize}
  \item Alto:
  \item Ancho:
  \item Profundidad:
 \end{itemize}
 \todo{Faltan las medidas}
 Sobre la estructura de la planta, van montado los diferentes elementos, en su sección 
 media presenta unos refuerzos, los cuales son necesarios dada la altura y peso de la misma, en 
 los cuales se fijo una base de madera donde se apoyan la válvula, las celdas de presión diferencial
 y las bombas. La manguera que corresponde al tiempo muerto se colocaron debajo de la base de madera.
 
\section{Cañerías}
\label{sec:Canerias}

\subsection{Mangueras y Caños}

\subsection{Válvulas manuales}
valvula que limita la entrada al reservorio
\subsection{Tiempo Muerto}
Una mención especial vamos a dar al denominado tiempo muerto, que consta de \todo{Cuantos metros de
manguera?} manguera negra de 1/2 pulgada, colocados en serie en el sistema, con una válvula que lo
habilita, cuya función es la de alejar la acción de control del tanque  controlado, esto introduce 
perturbaciones al sistema aumentando la complejidad del sistema de control. En esta situación el agua
debe recorrer más metros evidentemente, se agregan pérdidas por rozamiento del fluido y las bombas 
son forzadas a trabajar más.

Esto fue previsto para ofrecer a las diferentes cátedras que van a trabajar con la planta muchos 
escenarios que se presentan en la industria.

\section{Bombas}
\label{sec:Bombas}
Descripción, imagen

\section{Tanques}
\label{sec:Tanques}

\section{Válvula Neumática}
\label{sec:ValvulaNeumatica}
Foto, ubicación, datasheet , descripción

\section{Instrumentos de Medición}
\label{sec:InstrumentosMedicion}
\subsection{Placa Orificio}
\label{subsec:PlacaOrificio}

\subsection{DP cell}
\label{subsec:DPCell}

\subsection{Manómetros}
\label{subsec:Manometros}
Uno de los dos es viscoso, por qué? 
