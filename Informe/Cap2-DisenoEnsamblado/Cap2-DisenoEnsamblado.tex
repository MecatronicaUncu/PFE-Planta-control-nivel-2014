\chapter{Diseño y Ensamblado}
\label{ch:DisenoEnsamblado}

\section{P \& I D}
\label{sec:p&id}

\subsection{Diagrama de cañerías e instrumentación}
Un diagrama de cañerías e instrumentación es un esquema de la planta que nos brinda una idea 
general del problema que debemos abordar. En él se encuentra los elementos que componen el proceso. 
Estos diagramas emplean círculos y símbolos para representar cada elemento y como estan interconectados, 
los elementos presentes en este esquema están normalizados por la \emph{Standard S5.1
Instrumentation Symbol Specification}.

Cada elemento debe ir acompañado de una denominación la que presenta la siguiente codificación:


\begin{itemize}  
 \item Primer letra: 
 Es usada para designar la variable medida. Puede ser :
 \begin{itemize}
  \item Presión
  \item Nivel
  \item Caudal
  \item Temperatura
 \end{itemize}

 \item Letras siguientes:
 Son usadas para designar la función del componente, o para modificar el sentido de la primer letra.
 Pueden ser:
 \begin{itemize}
  \item Indicador
  \item Almacenaje de datos
  \item Controlador
  \item Transmisor
 \end{itemize}
\end{itemize}

\subsection{P \& I D de nuestra planta}
De acuerdo a los componentes detallados en el capitulo 1 de este informe se realizó el esquema P&ID


\section{Estructura de Soporte}
\label{sec:EstructuraSoporte}
Descripción mecánica

\section{Cañerías}
\label{sec:Canerias}
\subsection{Mangueras y Caños}
\subsection{Válvulas manuales}

\section{Bombas}
\label{sec:Bombas}
Descripción, imagen

\section{Tanques}
\label{sec:Tanques}

\section{Válvula Neumática}
\label{sec:ValvulaNeumatica}
Foto, ubicación, datasheet , descripción

\section{Instrumentos de Medición}
\label{sec:InstrumentosMedicion}
\subsection{Placa Orificio}
\label{subsec:PlacaOrificio}

\subsection{DP cell}
\label{subsec:DPCell}

\subsection{Manómetros}
\label{subsec:Manometros}
Uno de los dos es viscoso, por qué? 
