\chapter{Diseño y Ensamblado}
\label{ch:DisenoEnsamblado}
En esta sección intentaremos describir el camino realizado para la concreción del proyecto, desde 
las primeras instancias con la confección e interpretación del diagrama de cañerías e instrumentación, 
luego con la construcción de la planta a partir de los elementos disponibles, haciendo mención de 
las decisiones trascendentes para el trabajo.

\section{P \& I D}
\label{sec:p&id}

\subsection{Diagrama de cañerías e instrumentación}
Un diagrama de cañerías e instrumentación es un esquema de la planta que nos brinda una idea 
general del problema que debemos abordar. En él se encuentra los elementos que componen el proceso. 
Estos diagramas emplean círculos y símbolos para representar cada elemento y como están interconectados, 
los elementos presentes en este esquema están normalizados por la \emph{Standard S5.1
Instrumentation Symbol Specification}.

Cada elemento debe ir acompañado de una denominación la que presenta la siguiente codificación:


\begin{itemize}  
 \item Primer letra: 
 Es usada para designar la variable medida. Puede ser :
 \begin{itemize}
  \item Presión
  \item Nivel
  \item Caudal
  \item Temperatura
 \end{itemize}

 \item Letras siguientes:
 Son usadas para designar la función del componente, o para modificar el sentido de la primer letra.
 Pueden ser:
 \begin{itemize}
  \item Indicador
  \item Almacenaje de datos
  \item Controlador
  \item Transmisor
 \end{itemize}
\end{itemize}

\subsection{P \& I D de nuestra planta}
De acuerdo a los componentes detallados en el capitulo 1 de este informe se realizó el esquema P \& I D
de la planta de control de nivel. Para la confección de la misma se utilizó una plantilla disponible
en Google Docs, la decisión sobre la misma se hizo dado que era suficiente para la complejidad de
nuestro proyecto.
\todo{Colocamos la imagen acá o hacemos una referencia al anexo?}

\section{Estructura de Soporte}
\label{sec:EstructuraSoporte}

Dado que el objetivo principal del proyecto es hacer un elemento para fines educativos, la estructura
debía ser móvil, dotada de cuatro ruedas con frenos de seguridad que permitan mantener fija la estructura
al momento de la demostración de la planta.

La estructura es de caño estructural \todo{imagen de la estructura} y se puede apreciar en la figura
(tanto).
Sus dimensiones son las siguientes:

 \begin{itemize}
  \item Alto:
  \item Ancho:
  \item Profundidad:
 \end{itemize}
 \todo{Faltan las medidas}
 Sobre la estructura de la planta, van montado los diferentes elementos, en su sección 
 media presenta unos refuerzos, los cuales son necesarios dada la altura y peso de la misma, en 
 los cuales se fijo una base de madera donde se apoyan la válvula, las celdas de presión diferencial
 y las bombas. La manguera que corresponde al tiempo muerto se colocaron debajo de la base de madera.
 
 Al comienzo del proyecto la estructura de caño estructural estaba ya soldada y pintada, con las ruedas
 colocadas. Este grupo de trabajo continuo con la confección de la misma barnizando las maderas para la
 base.
\section{Montaje de los elementos}
En las próximas secciones de este informe se explicara en detalle la solución y la función de las diferentes
partes de la planta. En esta sección vamos a comentar el montaje de los diferentes elementos, las 
opciones con las que se contaba y decisiones que se tomaron al respecto. 

\section{Cañerías}
\label{sec:Canerias}

\subsection{Mangueras y Caños}
Las conexiones fueron echas en caño de polipropileno de 3/4 pulgadas y manguera de alta presión telada
de 3/4 pulgada, suficiente para las presiones con las que vamos a trabajar del orden de \todo{Con que
presión trabaja la planta?}. 

Luego de analizar el esquema P \& I D y  la estructura, se realizó la distribución de los elementos.
Para este fin se tuvo en cuenta la necesidad de mostrar claramente la función y posición de cada uno,
y ello requiere que todos los elementos y las cañerías se coloquen de manera que el alumno entienda
el proceso y sus componentes. El camino del  agua debe ser desde el tanque reservorio hacia el tanque
controlado y el regreso desde este hacia el tanque reservorio.

En este punto del proyecto tuvimos que tomar la decisión de como realizar el conexionado, ya teníamos 
montados los elementos, bombas y válvula neumática y decidido el espacio para las celdas de presión
diferencial.

El caño rígido presenta la dificultad al momento de realizar las conexiones de que no tenes tolerancia
con las medidas, además de que debes colocar más componentes, como uniones dobles, haciendo que las
cañerías se vean muy cargadas de elementos y de esta manera se pierde el objetivo principal que era 
ser lo  más claro posible para el alumno. Es por ello que se opto por usar en diferentes partes de 
la cañería mangueras flexibles.
A continuación vamos a describir las diferentes partes de la planta y la solución optada:

\begin{itemize}
  \item Salida del tanque reservorio, entrada a la bomba de impulsión:
  Se utilizo manguera dado que la estructura no permitía colocar caños rígidos, los elementos de 
  polipropileno necesarios eran más grandes de lo que nos permitía la estructura.
 
  \item Conexión de perturbación de la bomba de impulsión:
  La solución adoptada fue usar manguera para evitar uniones dobles y la realización de niples
  extremadamente precisos.
  
  \item Salida de la bomba, entrada a la válvula:
  Se utilizo manguera para evitar usar una unión doble en un trayecto que era muy pequeño.
  
  \item Salida de la válvula, entrada al tanque controlado
  Por la longitud de la misma se opto usar caños rígidos, de esta manera se puede montar los elementos
  de medición sobre la cañería. Sobre esta se montaron un manómetro y una placa orificio para ser capaces
  de medir caudal y conocer la situación de la planta.
  
  \item Salida del tanque controlado, entrada a la bomba de retorno:
  Debido a los mismo problemas de espacio con la estructura se opto por el uso de manguera flexible.
 
  \item Conexión de perturbación de la bomba de retorno:
  Se utilizo manguera para evitar usar una unión doble en un trayecto que era muy pequeño.
  
  \item Salida de la bomba, entrada al tanque reservorio
  Esta conexión es la más larga y sobre la misma van montados manómetros y válvulas manuales. Por ello
  se eligió una cañería rígida de caño de polipropileno.
  
 \end{itemize}

Por ultimo se procedió a pintar los caños de diferentes colores para ser capaces de diferenciar el 
camino del agua que llena el tanque controlado con el de regreso.

 \begin{itemize}
  \item Celeste: Camino de llenado del tanque.
  \item Amarillo: Camino de regreso.
 \end{itemize}
\todo{bella imagen de los colores de la cañería, fer agradece que no puse q vos sos el q las pinto}

\subsection{Válvulas manuales}
válvula que limita la entrada al reservorio
\subsection{Tiempo Muerto}
Una mención especial vamos a dar al denominado tiempo muerto, que consta de \todo{Cuantos metros de
manguera?} manguera negra de 1/2 pulgada, colocados en serie en el sistema, con una válvula que lo
habilita, cuya función es la de alejar la acción de control del tanque  controlado, esto introduce 
perturbaciones al sistema aumentando la complejidad del sistema de control. En esta situación el agua
debe recorrer más metros evidentemente, se agregan pérdidas por rozamiento del fluido y las bombas 
son forzadas a trabajar más.

Esto fue previsto para ofrecer a las diferentes cátedras que van a trabajar con la planta muchos 
escenarios que se presentan en la industria.

\section{Bombas}
\label{sec:Bombas}
Descripción, imagen

\section{Tanques}
\label{sec:Tanques}

\section{Válvula Neumática}
\label{sec:ValvulaNeumatica}
Foto, ubicación, datasheet , descripción

\section{Instrumentos de Medición}
\label{sec:InstrumentosMedicion}
\subsection{Placa Orificio}
\label{subsec:PlacaOrificio}

\subsection{DP cell}
\label{subsec:DPCell}

\subsection{Manómetros}
\label{subsec:Manometros}
Uno de los dos es viscoso, por qué? 
