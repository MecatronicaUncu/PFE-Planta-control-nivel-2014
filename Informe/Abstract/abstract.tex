\markboth{}{}
\pagestyle{empty}
%\vspace{-2cm}
%---------------------------%

\chapter*{Abstract}

En gran número de procesos industriales, se requiere controlar el caudal o el
nivel de un determinado fluido en un tanque o una cañería.
Para ello, se utilizan sensores capaces de medir la variable a
controlar, se ejecutan algoritmos de control en autómatas digitales y se
modifica el estado del proceso mediante actuadores (válvulas, variadores de
velocidad).
Conocer como realizar el control de caudal y nivel de una planta, en todas sus
partes (sensores, actuadores y programación del autómata) es un requisito
indispensable para los futuros ingenieros Mecatrónicos, de Petróleos e
Industriales.

En este \gls{pfe}, se diseñará y construirá una planta piloto de control de
nivel.
Se utilizarán como elementos de adquisición sensores de caudal (DP cell y placa
orificio) y de nivel de tanque (DP cell).
El actuador será una válvula electroneumática.
Todo el proceso será controlado por un \gls{plc}, que ejecutará las acciones de
corrección necesarias para una consigna de caudal o nivel dado\todo{caudal
también?}.
Además, el \gls{plc} se comunicará con una computadora de control que ejecuta
un programa de tipo \gls{scada} con el propósito de observar el estado de la
planta, como así también enviar nuevas consignas de nivel o
caudal\todo{caudal?}.

Una característica distintiva de este trabajo es que la planta será móvil.
Es decir, se encuentra montada sobre una estructura con ruedas para permitir su
libre desplazamiento.
Además, la conexión con la computadora de control se realizará de manera
inalámbrica.

En este informe de proyecto se describen el proceso de construcción,
ensamblado y conexionado de la planta.
Además se realizará la programación del \gls{plc} y del \gls{scada}, con el
propósito de verificar el correcto funcionamiento de la planta.
Se entregarán junto con este informe el software, circuitos y diagramas para
ser utilizado durante los trabajos prácticos o en futuros proyectos.
