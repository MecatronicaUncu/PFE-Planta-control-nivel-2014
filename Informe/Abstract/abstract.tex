\markboth{}{}
\pagestyle{empty}
%\vspace{-2cm}
%---------------------------%

\chapter*{Abstract}

En gran número de procesos industriales, se requiere controlar el
nivel de un determinado fluido en un tanque.
Para lograr esta tarea, se utilizan en la actualidad esquemas de control
basados en computadores digitales.
Mediante sensores se mide la variable a
controlar, se ejecutan algoritmos de control en autómatas y se
modifica el estado del proceso mediante actuadores (válvulas, variadores de
velocidad, etc.).
Conocer los detalles del control de nivel en un proceso industrial, en todas sus
partes, es un requisito
indispensable para cualquier ingeniero en Mecatrónica, Industrial o de
Petróleos.

En este \gls{pfe}, se diseñará y construirá una planta piloto de control de
nivel.
Se utilizarán como elementos de adquisición sensores de caudal (DP cell y placa
orificio) y de nivel de tanque (DP cell).
El actuador será una válvula electroneumática.
Todo el proceso será controlado por un \gls{plc},
que ejecutará las acciones de
corrección necesarias para obtener una consigna de nivel objetivo.
Además, el \gls{plc} se comunicará con una computadora de control que ejecuta
un programa de tipo \gls{scada}.
Se podrá observar el estado de la
planta de manera remota, como así también enviar nuevas consignas de nivel,
entre otras.

Una característica distintiva de este trabajo es que la planta será móvil.
Es decir, se encuentra montada sobre una estructura con ruedas para permitir su
libre desplazamiento.
Además, la conexión con la computadora de control se realizará de manera
inalámbrica.

En este informe de proyecto se describen el proceso de construcción,
ensamblado y conexionado de los diferentes elementos constitutivos la planta.
Se realizará la programación del \gls{plc} y del \gls{scada}, con el
propósito de verificar su correcto funcionamiento.
Se entregarán junto con este informe el software, circuitos y diagramas para
que sean utilizados en trabajos prácticos o en futuros proyectos.

% REVISIÓN 1 - fclad
