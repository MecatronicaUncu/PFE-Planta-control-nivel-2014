\markboth{}{}
\pagestyle{empty}
%\vspace{-2cm}
%---------------------------%

\chapter*{Abstract}

En gran número de procesos industriales, se requiere controlar el
nivel de un determinado fluido en un tanque.
Para lograr esta tarea, se utilizan en la actualidad esquemas de control
basados en computadores digitales.
Mediante transmisores de señal se mide la variable a
controlar, se ejecutan algoritmos de control en autómatas y se
modifica el estado del proceso mediante actuadores (válvulas, variadores de
velocidad, etc.).
Conocer los detalles del control de nivel en un proceso industrial, en todas sus
partes, es un requisito
indispensable para cualquier ingeniero en Mecatrónica, Industrial o de
Petróleos.

En este \gls{pfe}, se diseñó, se construyó y se puso en marcha una planta
piloto de control de nivel.
Se utilizaron como elementos de adquisición transmisores de caudal (DP cell
y placa
orificio) y de nivel de tanque (DP cell).
El actuador es una válvula globo, con servoactuador de diafragma -
resorte y electroposionador neumático.
Todo el proceso es controlado por un \gls{plc},
que ejecuta las acciones de
corrección necesarias para obtener un control óptimo del nivel, tanto lo
requerido ante cambios del punto de consigna o ante perturbaciones.
Además, el \gls{plc} se comunica con una computadora supervisora que ejecuta
un programa de tipo \gls{scada}.
Se puede observar el estado de la
planta de manera remota, como así también pueden enviarse nuevas consignas de
nivel,
entre otras.

Una característica distintiva de este trabajo es que la planta es móvil.
Es decir, se encuentra montada sobre una estructura con ruedas para permitir su
libre desplazamiento.
Además, la conexión con la computadora supervisora se realiza de manera
inalámbrica.

En este informe de proyecto se describen el proceso de construcción,
ensamblado y conexionado de los diferentes elementos constitutivos la planta.
Se analiza la programación del \gls{plc} y del \gls{scada}, con el
propósito de verificar el correcto funcionamiento de la planta.
Se entregan junto con este informe el software, circuitos y diagramas para
que sean utilizados en trabajos prácticos o en futuros proyectos.

% REVISIÓN 1 - fclad
% CORRECCION 1 Puglesi - fclad
