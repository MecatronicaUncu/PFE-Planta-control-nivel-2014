\chapter{SCADA}
\label{ch:scada}

Como elemento final del \gls{pfe} se diseño y se realizo una \gls{hmi} \gls{scada}
mediante P-CIM para Windows.

\section{Introducción a SCADA}
\label{sec:IntroScada}
Aunque un PLC realiza automáticamente un control pre-programado sobre un proceso, 
normalmente se distribuyen a lo largo de toda la planta, haciendo difícil recoger 
los datos de manera manual, los sistemas SCADA lo hacen de manera automática. 
Históricamente los PLC no tienen una manera estándar de presentar la información 
al operador. La obtención de los datos por el sistema SCADA parte desde el PLC o 
desde otros controladores y se realiza por medio de algún tipo de red, 
posteriormente esta información es combinada y formateada. Un HMI puede tener 
también vínculos con una base de datos para proporcionar las tendencias, los datos 
de diagnóstico y manejo de la información así como un cronograma de procedimientos 
de mantenimiento, información logística, esquemas detallados para un sensor o 
máquina en particular, incluso sistemas expertos con guía de resolución de 
problemas.

 le permite proporcionar alarmas integradas y monitoreo de eventos así como la 
 adquisición, análisis y presentación de la información de plantas, máquinas y 
 equipos industriales, ahora también factible de extender a la domótica.

P-CIM funciona en PC y se conecta generalmente a diferentes controladores 
lógicos Programables (PLCs) y otros dispositivos como controladores analógicos 
mono o multilazos como así también a otros periféricos inteligentes. 


Usando PCIM para Windows los operadores de planta conocen instantáneamente 
el estado de los procesos de la planta. Los datos en Tiempo Real y las 
tendencias históricas se pueden presentar en pantalla, permitiendo la 
determinación del estado del proceso al instante.

P-CIM para Windows le permite pre-configurar acciones a ser ejecutadas 
automáticamente como resultado de otras acciones o cuando se alcanzan 
ciertas condiciones específicas.

• Capa de Comunicación – Esta capa se encarga de la comunicación con los PLCs y redes.
• Capa de Procesamiento de Datos – Esta capa lleva a cabo la mayor parte del procesamiento de datos, registro histórico y manejo de alarmas.
• Capa de aplicación – Esta capa presenta la información, interactúa con el operador y realiza los controles de alto nivel y de programación.
\todo {dibujo de capas de abstraccion}

\subsection{Capa de comunicación}
\label{subsec:CapaComunicacion}

La capa de comunicación recibe información del campo a través del 
PLC, la transfiere al Servidor de la Base de Datos (Database Server) 
que analiza la información, la capa de aplicación la procesa y la 
envía hacia la pantalla (interfase con el Operador)

\section{Capas del programa}
\label{sec:CapasPrograma}



\subsection{Capa de procesamiento}
\label{subsec:CapaProcesamiento}


\subsection{Capa de aplicación}
\label{subsec:CapaAplicacion}


\section{Ejecución}
\label{sec:Ejecucion}
