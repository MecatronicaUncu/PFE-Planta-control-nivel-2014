\section{Contexto}
\frame{
	\frametitle{Outline}
	\tableofcontents
	}

\subsection*{Introducción}
% Una subsection* crea un nuevo "grupo de puntos"

\frame{
	\frametitle{Descripción del problema}

	\begin{center}
		\textbf{Definición}

		\vspace{0.25cm}

		\missingfigure[figwidth=6cm]{Figura completa}
	\end{center}

	\begin{columns}
		\begin{column}{0.5\textwidth}
			\begin{itemize}
				\item {\color{newcolor} Texto en columna 1}:
				\begin{itemize}
					\item item 1
					\item item 2
				\end{itemize}
				\item {\color{newcolor} Más texto}
			\end{itemize}
		\end{column}

		\begin{column}{0.5\textwidth}
			\begin{itemize}
				\item {\color{softBlue!70!Black} Texto:}
					en otra columna
				\item {\color{softOrange!70!Black} Más texto:}
					en otra columna
			\end{itemize}

		\end{column}
	\end{columns}
}

\frame{
	\frametitle{Especificaciones}
	\begin{columns}
		\begin{column}{0.4\textwidth}
			\begin{enumerate}
				\item Una
				\begin{itemize}
					\item \textbf{Larga}
					\item larga
				\end{itemize}
				\item Lista
			\end{enumerate}

			\begin{align*}
				\mathrm{una} \,e( \textbf{cuacion})
			\end{align*}
		\end{column}

		\begin{column}{0.6\textwidth}
			\begin{center}
				\missingfigure[figwidth=4.5cm]{}
			\end{center}

			\begin{center}
				\footnotesize
				\textbf{\textit{Cita bibliográfica}}

				[Escrita por alguien \textit{et al.} 2008]
			\end{center}
		\end{column}

	\end{columns}
}

\subsection*{Otra subsection}

\frame{
	\frametitle{Otro slide completo}
	\textbf{Idea:} ...
	\vspace{0.25cm}
	\begin{center}

		Lorem ipsum dolor sit amet, consectetur adipiscing elit.
		Donec viverra cursus pellentesque. In et mattis augue.
		Dorbi ut velit a ante ultrices ornare in id nisl.

		\vspace{0.5cm}

		Pellentesque sollicitudin bibendum leo lobortis sagittis.
		Fusce eu purus vel mauris vestibulum mattis.
	\end{center}
}
